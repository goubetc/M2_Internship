\documentclass[10pt,a4paper]{article}
\usepackage[utf8]{inputenc}
\usepackage{amsmath}
\usepackage{amsfonts}
\usepackage{amssymb}
\usepackage{graphicx}
\usepackage{cite}
\author{Claude Goubet}
\title{Short Review On Low Dose CT Reconstruction}
\begin{document}
\maketitle

\section{Introduction}
	intro about CT and importance for osteoporosis diagnosis + use of SR + low dose problem CS (\cite{[24], [25], [26]})

\section{Dose reduction in SR Micro-CT}
Multiple CS algorithm were developed for Micro-CT allowing to generate different spacial resolutions. Alternative methods then FBP necessary to recover missing projections. Iterative algorithms are used.
	\subsection{No SR}
		SART-L1 \cite{[11],[13]} ASD-POCS TV \cite{[9]}
	\subsection{CS on SR micro-CT}
		multiple iterative methods using CGTV (\cite{[12]}) ART with multiple denoising (TV \cite{[3]}; L1 minimisation \cite{[18]}; Discrete packet shrinkage \cite{[2]}) SART (\cite{[1]} with TV \cite{[5]}) OS-SART \cite{[6]}) EST \cite{[15], [16]}  PCCT \cite{[8]}
		 define resolution for each solution (maybe more details?)

\section{SR Nano-CT}
	Nano-CT general ref: \cite{[23]} (I can have other references but are mostly about the hardware side, new materials and acquisition methodology, or image post-processing without having used low dose)\\
	less CS reconstruction experimented	\\
	Low dose nano OS-SART L1 norm TV \cite{[10]}

\section{Conclusion}
	A lot of research these past few years of CSCT going toward a improvement of spacial resolution and dose reduction. Yet not so much has been done on Nano scale. In the context of osteoporosis nano scale is mandatory for a accurate diagnosis. Present our objective.
%\section{previous review}
%	\cite{[17]}
	
%\section{TV}
%Minimization of total variation (TV) is a well used algorithm on syncrothron CT. In this optimization-based approach a discrete imaging model described by a linear equation system in which the number of equations is determined by that of projection measurements.\\

%Han, X. et al. \cite{[9]} described the adaptive-steepest-descent-progection-onto-convex-sets (ADS-POCS) algorithm, which reconstructs an image through minimizing the image total-variation and enforcing data constraints. Using this algorithm they were able to produce image comparable to usual algorithms using only one sixth of the data. This algorithm was evaluated on a imaging experiments of porcine heart and kidney specimens acquired with a custom-made micro-CT scanner to collect cone-beam data in our specimen-imaging studies\\

%Liang, Z. et al \cite{[10]} imroved the conventional Fourier reconstruction method on nano-CT
%it is difficult to acquire nano-CT images with high quality by using conventional Fourier reconstruction methods based on limited-angle or few-view projections, utilized the total variation (TV) iterative reconstruction to carry out numerical image and nano-CT image reconstruction with limited-angle and few-view data. \\

% Yang, X et al. \cite{[12]} reduced the number of projections by applying an advanced algebraic technique (optimised CGTV) subject to the minimization of the TV in the reconstructed slice. This problem is formulated in a Lagrangian multiplier fashion with the parameter value determined by appealing to a discrete L-curve in conjunction with a conjugate gradient method. The optimal value of the multiplier is automatically obtained by imposing an independent L-curve criterion.
% The usefulness of this reconstruction modality is demonstrated for simulated and in vivo data, the latter acquired in parallel-beam imaging experiments using synchrotron radiation.
 % they were able to reduce the number of projections by 60–75\% \\

%Fahimian, B. P. et al. \cite{[15]} developed new reconstruction algorithms which allow for the incorporation of advanced mathematical regularization constraints, such as the non-local means total variational model, in a manner that is consistent with experimental projections. This method results in an iterative Fourier-based tomographic reconstruction algorithm which incorporates the advanced total variation implementation of the non-local means regularization, in a manner that is strictly consistent with experimentally measured projections. They implemented the method on the absorption and phase contrast x-ray CT modalities, and through a series of experiments and simulations quantify the resulting image quality. This method can reduce the number of projections by 60–75\% in parallel beam modalities, while achieving comparable or better image quality than the conventional reconstructions.
%\\

%\section{SART/ART}
%\cite{[1],[6],[18]}

%\section{L1}
%\cite{[19]}

%\section{Hybrid}
%\subsection{SART+TV}
%	\cite{[5]}
%\subsection{ART+L1}
%	\cite{[11],[13]}
%\subsection{TV+L1}
%	\cite{[14]}
%\subsection{OS-SART}
%	\cite{[6]}
%\section{Douglas-Rachford}
%	\subsection{packet shrinkage denoising}
%	\cite{[2]}
%	\subsection{randomized Kaczmarz algorithms}
%	\cite{[3]}
%\section{PBI}
%	\cite{[4]}
%\section{PPSST}
%	\cite{[16]}
%\section{others}
%	\cite{[7],[8]}


	

\bibliography{biblio.bib}{}
\bibliographystyle{plain}
\end{document}