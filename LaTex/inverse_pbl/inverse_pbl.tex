\documentclass[10pt,a4paper]{report}
\usepackage[utf8]{inputenc}
\usepackage{amsmath}
\usepackage{amsfonts}
\usepackage{amssymb}
\usepackage{graphicx}
\author{Claude Goubet - Juan JFP Abascal}
\title{Introduction to inverse problem resolution}
\begin{document}
\maketitle
\chapter{Inverse problem and conditioning}

\section{Inverse problem}
An inverse problem is a situation when you wish to describe a physical model from the results of measurements. As apposed to modelization problem where you estimate the outcome of measurements using the physical model.\\
Basically, a modelization problem is of the form:
\begin{equation}
	A \times x = Y
\label{ForwardPbl}
\end{equation}
Where you estimate a measurement $Y$ of the estimated measurement of element $x$ with a transformation matrix $A$. Hence the known values are $x$ and $A$ and compute $Y$.\\
The inverse problem is posed as in equation (\ref{ForwardPbl}) although the known values are the measurement $Y$ and the transformation matrix $A$ and we want to infer from them the physical model $x$.\\
A naive solution to (\ref{ForwardPbl}) in case of inverse problem would be:
\begin{equation}
	x = A^{-1} \times Y
\end{equation}
Thus this solution requires $A^{-1}$ to be invertable such transformation requires a the problem to be well posed. In the contrary case, other methods must be used in order to avoid using $A^{-1}$.

\section{well- and ill-posed problems}

For the following, let us consider $A$ a squared matrix of size $n \times n$; $x$ of the size $n \times 1$ and $Y$ of the size $n \times 1$.\\

In order for a matrix to be invertable, its rank must be equal to its number of columns. The rank of a matrix being its number of independent rows. 


\end{document}

